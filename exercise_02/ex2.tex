
%% GENERAL DEFINITIONS
\documentclass{article} %% Determines the general format.
\usepackage{a4wide} %% paper size: A4.
\usepackage[utf8]{inputenc} %% If problem with special characters: "latin1"
\usepackage[T1]{fontenc} %% Format of the resulting PDF file.
\usepackage{fancyhdr} %% Package to create a header on each page.
\usepackage{lastpage} %% Used for "Page X of Y" in the header.
\usepackage{enumerate} %% Used to change the style of enumerations (see below).
\usepackage{amssymb} %% Definitions for math symbols.
\usepackage{amsmath} %% Definitions for math symbols.
\usepackage{amsthm} %% === Definition for proof paragraph ===
\usepackage{color}
\usepackage{comment}

%% NICE ADDITIONS
% \usepackage{parskip} %% uses empty lines for paragraphs instead of \
% \usepackage{MnSymbol,wasysym} % smileys: \smiley{} \frownie{} \blacksmiley{}

%% Remove initial indentation of first line of paragraphs
\setlength{\parindent}{0pt}

%% GRAPHS
\usepackage{tikz}
\usetikzlibrary{automata}
%% Tikz library for nicer arrow heads
\usetikzlibrary{calc,trees,positioning,arrows,fit,shapes,calc} 

%% CODE LISTINGS
\usepackage{listings}

\lstset{
    breaklines=true,
    inputpath=code, 
    language=C, 
    basicstyle=\fontencoding{T1}\small\fontfamily{lmtt}\fontseries{m}\selectfont, 
    keepspaces=true, 
    numbers=left, 
    stepnumber=1, 
    frame=shadowbox, 
    rulesepcolor=\color{gray}, 
    commentstyle=\color{dkgreen}, 
    keywordstyle=\color{blue},
    showstringspaces=false}

%% DEFINITIONS
\definecolor{dkgreen}{rgb}{0,0.6,0}
\definecolor{gray}{rgb}{0.5,0.5,0.5}
\definecolor{mauve}{rgb}{0.58,0,0.82}

\newtheorem{statement}{Statement}

%% PATHS
\graphicspath{{images/}} %% adds graphics path

%%%%%%%%%%%%%%%%%%%%%%%%%%%%%%%%%%%%%%%%%%%%%%%%%%%%%%%%%%%%%%%%%%%%%%%%%%%%%%%%
%%%%%%%%%%%%%%%%%%%%%%%%%%%%%%%%%%%% HEADER %%%%%%%%%%%%%%%%%%%%%%%%%%%%%%%%%%%%
%%%%%%%%%%%%%%%%%%%%%%%%%%%%%%%%%%%%%%%%%%%%%%%%%%%%%%%%%%%%%%%%%%%%%%%%%%%%%%%%
%% Left side of header
\lhead{\course\\\semester\\Exercise Sheet \homeworkNumber}
%% Right side of header
\rhead{\authorname\\Page \thepage\ of \pageref{LastPage}}
%% Height of header
\usepackage[headheight=36pt]{geometry}
%% Page style that uses the header
\pagestyle{fancy}
%% add title to header
%% \chead{\huge{\textbf{Blatt \homeworkNumber}}}

%%%%%%%%%%%%%%%%%%%%%%%%%%%%%%%%%%%%%%%%%%%%%%%%%%%%%%%%%%%%%%%%%%%%%%%%%%%%%%%%
%%%%%%%%%%%%%%%%%%%%%%%%%%%%%%%%%%% METADATA %%%%%%%%%%%%%%%%%%%%%%%%%%%%%%%%%%%
%%%%%%%%%%%%%%%%%%%%%%%%%%%%%%%%%%%%%%%%%%%%%%%%%%%%%%%%%%%%%%%%%%%%%%%%%%%%%%%%
\newcommand{\authorname}{Nico Bachmann, Valentin Neher} % TODO: add your names
\newcommand{\semester}{Fall Term 2023}
\newcommand{\course}{Computer Architecture}
\newcommand{\homeworkNumber}{2} % TODO: add current sheet nr


\begin{document}
%%%%%%%%%%%%%%%%%%%%%%%%%%%%%%%%%%%%%%%%%%%%%%%%%%%%%%%%%%%%%%%%%%%%%%%%%%%%%%%%
\section*{Exercise \homeworkNumber.1}

\begin{flushright}
\textbf{ex2-1.c}
\lstinputlisting{ex2-1.c}
\end{flushright}
We compile the program with: \verb|gcc -o ex2-1 ex2-1.c| and receive the compiled executable \verb|ex2-1| which we can run using \verb|./ex2-1|. \\

Console Output:\\
\verb|Array in reverse Order using [ ] : 9, 8, 7, 6, 5, 4, 3, 2, 1, 0, | \\
\verb|Array in reverse Order using a pointer: 9, 8, 7, 6, 5, 4, 3, 2, 1, 0, |

\newpage

%%%%%%%%%%%%%%%%%%%%%%%%%%%%%%%%%%%%%%%%%%%%%%%%%%%%%%%%%%%%%%%%%%%%%%%%%%%%%%%
\section*{Exercise \homeworkNumber.2}
\begin{flushright}
\textbf{ex2-2.c}
\lstinputlisting{ex2-2.c}
\end{flushright}
We compile the program with: \verb|gcc -o ex2-2 ex2-2.c| and receive the compiled executable \verb|ex2-2| which we can run using \verb|./ex2-2|. \\

Console Output:\\
\verb|(2, 7, 6)|

\newpage
%%%%%%%%%%%%%%%%%%%%%%%%%%%%%%%%%%%%%%%%%%%%%%%%%%%%%%%%%%%%%%%%%%%%%%%%%%%%%%%%

\section*{Exercise \homeworkNumber.3}
First we include \verb|stdio.h| and \verb|stdlib.h| to get access to \verb|printf()|, \verb|malloc()| and \verb|free()|. Then we define our \verb|Vector| struct. We could also include it from the last exercise but then the \emph{two} main functions will do trouble. For the linked list we need a \verb|Node| structure to hold the data and a pointer to the next Node and a List Structure (\verb|VectorList|) that holds pointers to the head of the list and optionally also to the tail.\\

Additionally we write a short helper function \verb|printVec| that helps us to easily print our vectors.\\

Because the \verb|size()| method is used before it is defined, we write this prototype here to let the compiler know of its existence.

\begin{flushright}
\textbf{ex2-3.c}
\lstinputlisting[firstline=1, lastline=30, firstnumber=1]{ex2-3.c}
\end{flushright}
\newpage
%%%%%%%%%%%%%%%%%%%%%%%%%%%%%%%%%%%%%%%%%%%%%%%%%%%%%%%%%%%%%%%%%%%%%%%%%%%%%%%%

Now we're writing the first method \verb|getElement()| which returns a pointer to the \verb|Vector| at the given \verb|index| in the \verb|VectorList|. To do that, we first check that the given \verb|index| is not out of bounds. Secondly we handle the case when the \verb|VectorList| is empty. With those edge-cases handled, we either return \verb|head_ptr->vector| when the index is 0 or we iterate through the list to the requested \verb|index|.

\begin{flushright}
\textbf{ex2-3.c}
\lstinputlisting[firstline=32, lastline=58, firstnumber=32]{ex2-3.c}
\end{flushright}
\newpage
%%%%%%%%%%%%%%%%%%%%%%%%%%%%%%%%%%%%%%%%%%%%%%%%%%%%%%%%%%%%%%%%%%%%%%%%%%%%%%%%

The \verb|insertElementFront()| method is supposed to add an element to the front of the \verb|Vectorlist|. Since the list should be stored on the heap, we need to allocate space for a new \verb|Node| where we can store the \verb|Vector| pointer in. We initialize a temporary pointer to point at this new \verb|Node| in the heap, store the \verb|vector| pointer inside it and let it point to the head of the \verb|VectorList|. Therefore making it the new head. We change the \verb|head_ptr| which currently points to the second element to now point to the first. Lastly we take care of an edge-case, if the \verb|VectorList| was empty before, its \verb|tail_ptr| also has to point to our newly added \verb|Node| because with only a single element, tail and head are the same.
\begin{flushright}
\textbf{ex2-3.c}
\lstinputlisting[firstline=62, lastline=79, firstnumber=62]{ex2-3.c}
\end{flushright}
\newpage
%%%%%%%%%%%%%%%%%%%%%%%%%%%%%%%%%%%%%%%%%%%%%%%%%%%%%%%%%%%%%%%%%%%%%%%%%%%%%%%%

The next method does basically the same, just this time adding the element to the back instead of the front. We again create a \verb|Node| on the heap with a temporary pointer pointing to it. This time tough the new \verb|Node| points to \verb|NULL| since it is the last \verb|Node|. 

We then handle the case of the empty \verb|VectorList| where our last \verb|Node| is also the first one. We therefore let the \verb|head_ptr| point to it. On the other hand, if the \verb|VectorList| was not empty, its current last \verb|Node| needs to point to our new node. And finally we also let the \verb|tail_ptr| point to our new last \verb|Node|.

\begin{flushright}
\textbf{ex2-3.c}
\lstinputlisting[firstline=81, lastline=102, firstnumber=81]{ex2-3.c}
\end{flushright}

The \verb|size()| method simply iterates through the list until the current \verb|Node| is \verb|NULL| and then returns the counter.

\begin{flushright}
\textbf{ex2-3.c}
\lstinputlisting[firstline=104, lastline=113, firstnumber=104]{ex2-3.c}
\end{flushright}
\newpage
%%%%%%%%%%%%%%%%%%%%%%%%%%%%%%%%%%%%%%%%%%%%%%%%%%%%%%%%%%%%%%%%%%%%%%%%%%%%%%%%

In the \verb|main()| method we create a \verb|VectorList| and a pointer to it and some \verb|Vectors|, also with pointers to them. We're then inserting the \verb|Vectors| in our \verb|VectorList| using the two different methods and then print out its size as well as the elements using the \verb|getElement()| method.
\begin{flushright}
\textbf{ex2-3.c}
\lstinputlisting[firstline=115, lastline=144, firstnumber=115]{ex2-3.c}
\end{flushright}

We compile the program with: \verb|gcc -o ex2-3 ex2-3.c| and receive the compiled executable \verb|ex2-2| which we can run using \verb|./ex2-3|. Console Output:\\
\verb|Size of the list is 0|\\
\verb|Size of the list is 4|\\
\verb|First element is:|\\
\verb|(1, 1, 1)|\\
\verb|Second element is:|\\
\verb|(2, 2, 2)|\\
\verb|Third element is:|\\
\verb|(3, 3, 3)|\\
\verb|Fourth element is:|\\
\verb|(4, 4, 4)|\\

\newpage

%%%%%%%%%%%%%%%%%%%%%%%%%%%%%%%%%%%%%%%%%%%%%%%%%%%%%%%%%%%%%%%%%%%%%%%%%%%%%%%%
\section*{Exercise \homeworkNumber.4}

Just like before we create our \verb|Vector| struct, a helper function to print the vector, another one to print an array of vectors and a third one to calculate the \verb|L2 form| of a vector.

Normally we could just \verb|#include| those things from the other files but since every file has its own \verb|main()| method for that particular exercise task, the compiler loses his mind and does not want to work with multiple \verb|main()| methods...

\begin{flushright}
\textbf{ex2-4.c}
\lstinputlisting[firstline=1, lastline=33, firstnumber=1]{ex2-4.c}
\end{flushright}
\newpage

%%%%%%%%%%%%%%%%%%%%%%%%%%%%%%%%%%%%%%%%%%%%%%%%%%%%%%%%%%%%%%%%%%%%%%%%%%%%%%%%

For the \verb|qsort()| method to work, we need to provide a way to compare our \verb|Vectors|. Apparently as parameters, "raw?" pointers should be used and only inside the function we cast them to be \verb|Vector| pointers.

We then compare the two \verb|Vectors| by their \verb|L2 form| to decide if the first one is "greater", "even", or "lesser" than the second one.

\begin{flushright}
\textbf{ex2-4.c}
\lstinputlisting[firstline=35, lastline=47, firstnumber=35]{ex2-4.c}
\end{flushright}

Lastly we write a method to print our array of \verb|Vectors| to a file. We open the file using the \verb|fopen()| method, check if it really opened, then write to it using \verb|fprintf()|. When we're finished, we \verb|close| the file.

\begin{flushright}
\textbf{ex2-4.c}
\lstinputlisting[firstline=49, lastline=66, firstnumber=49]{ex2-4.c}
\end{flushright}
\newpage


%%%%%%%%%%%%%%%%%%%%%%%%%%%%%%%%%%%%%%%%%%%%%%%%%%%%%%%%%%%%%%%%%%%%%%%%%%%%%%%%

In our \verb|main()| method we initialize our array using a for loop, then print it out, sort it with our \verb|compareVec()| method and print it out again. At the end, we write our result to the file using our \verb|writeToFile()| method.

\begin{flushright}
\textbf{ex2-4.c}
\lstinputlisting[firstline=68, lastline=85, firstnumber=68]{ex2-4.c}
\end{flushright}


\end{document}